% Options for packages loaded elsewhere
\PassOptionsToPackage{unicode}{hyperref}
\PassOptionsToPackage{hyphens}{url}
%
\documentclass[
]{article}
\usepackage{lmodern}
\usepackage{amssymb,amsmath}
\usepackage{ifxetex,ifluatex}
\ifnum 0\ifxetex 1\fi\ifluatex 1\fi=0 % if pdftex
  \usepackage[T1]{fontenc}
  \usepackage[utf8]{inputenc}
  \usepackage{textcomp} % provide euro and other symbols
\else % if luatex or xetex
  \usepackage{unicode-math}
  \defaultfontfeatures{Scale=MatchLowercase}
  \defaultfontfeatures[\rmfamily]{Ligatures=TeX,Scale=1}
\fi
% Use upquote if available, for straight quotes in verbatim environments
\IfFileExists{upquote.sty}{\usepackage{upquote}}{}
\IfFileExists{microtype.sty}{% use microtype if available
  \usepackage[]{microtype}
  \UseMicrotypeSet[protrusion]{basicmath} % disable protrusion for tt fonts
}{}
\makeatletter
\@ifundefined{KOMAClassName}{% if non-KOMA class
  \IfFileExists{parskip.sty}{%
    \usepackage{parskip}
  }{% else
    \setlength{\parindent}{0pt}
    \setlength{\parskip}{6pt plus 2pt minus 1pt}}
}{% if KOMA class
  \KOMAoptions{parskip=half}}
\makeatother
\usepackage{xcolor}
\IfFileExists{xurl.sty}{\usepackage{xurl}}{} % add URL line breaks if available
\IfFileExists{bookmark.sty}{\usepackage{bookmark}}{\usepackage{hyperref}}
\hypersetup{
  pdftitle={Week 8, Day 2},
  hidelinks,
  pdfcreator={LaTeX via pandoc}}
\urlstyle{same} % disable monospaced font for URLs
\usepackage[margin=1in]{geometry}
\usepackage{graphicx,grffile}
\makeatletter
\def\maxwidth{\ifdim\Gin@nat@width>\linewidth\linewidth\else\Gin@nat@width\fi}
\def\maxheight{\ifdim\Gin@nat@height>\textheight\textheight\else\Gin@nat@height\fi}
\makeatother
% Scale images if necessary, so that they will not overflow the page
% margins by default, and it is still possible to overwrite the defaults
% using explicit options in \includegraphics[width, height, ...]{}
\setkeys{Gin}{width=\maxwidth,height=\maxheight,keepaspectratio}
% Set default figure placement to htbp
\makeatletter
\def\fps@figure{htbp}
\makeatother
\setlength{\emergencystretch}{3em} % prevent overfull lines
\providecommand{\tightlist}{%
  \setlength{\itemsep}{0pt}\setlength{\parskip}{0pt}}
\setcounter{secnumdepth}{-\maxdimen} % remove section numbering

\title{Week 8, Day 2}
\author{}
\date{\vspace{-2.5em}}

\begin{document}
\maketitle

Weeks 7, 8 and 9 are the core of the class. We have a question. We have
some data. How should we use the data to answer the question? Using
Wisdom, we first decide if the question and the data are ``close
enough'' that we can consider them to both be part of the same
population. With Justice, we create a mathematical model which describes
the connection between the outcome we want to explain/understand an the
covariates which might be connected to it. Courage takes us from
mathematics to code, creating a model, including posterior distributions
for all its parameters. The last step is to use that model to answer the
question with which we started, with Temperance.

\textbf{Question:} \emph{What would Preceptor's change in attitude
toward immigration be if he were to receive the treatment next week?}

\hypertarget{scene-1}{%
\subsection{Scene 1}\label{scene-1}}

\textbf{Prompt:} We have not practiced writing mathematical formulas in
R markdown documents. Doing so is fairly simple. Bracket the math in one
dollar sign if you want it to appear inline. Example: a y in math
notation looks like \(y\) and is produced with \$y\$. Subscripts require
an underscore, and a bracket is often helpful. To make \(y_i\) you write
\$y\_i\$. For \(x_{t,i}\) write \$x\_\{t, i\}\$. To put the math on its
own line, use two dollar signs. Here is a version of one of the formulas
from chapter 8:

\[ age_i = \beta_1 x_{r,i} + \beta_2 x_{d,i} + \epsilon_i\] Today, we
will be considering \texttt{att\_chg} as a function of
\texttt{treatment}.

\begin{itemize}
\tightlist
\item
  Create the mathematical formula for that model here.
\end{itemize}

\[ att_chg_i \beta_1 x_[r, i] + \beta_2 x_[d, i} + \epsilon_i\]
-att\_chg is dependent variable -1 independent variable: treatment
-epsilon represents error

\begin{itemize}
\item
  Describe the ideal Preceptor Table.
\item
  Write a few sentences about whether or not our data will allow us to
  answer this question.
\end{itemize}

\hypertarget{scene-2}{%
\subsection{Scene 2}\label{scene-2}}

\textbf{Prompt:} There are two ways to interpret our question. The first
involves our expectation. We can also think of this as the long-term
average, if we did the same experiment over and over again. (Assume that
Preceptor is always assigned to the treatment.) What is your posterior
probability distribution for Preceptor's \emph{expected}
\texttt{att\_chg}?

\hypertarget{scene-3}{%
\subsection{Scene 3}\label{scene-3}}

\textbf{Prompt:} The second way to interpret the question is with regard
to a single individual. What will happen to Preceptor next week? Provide
a Posterior Predictive Distribution for what you predict will be
Preceptor's \texttt{att\_chg} after treatment.

\hypertarget{challenge-problem}{%
\subsection{Challenge Problem}\label{challenge-problem}}

Create the posterior distribution for the average treatment effect of
\texttt{treatment} within subsets of the data defined by
\texttt{liberal} and by \texttt{gender}. In other words, there should be
4 plots total: \texttt{liberal\ ==\ FALSE}, \texttt{liberal\ ==\ TRUE},
\texttt{gender\ ==\ "Male"} and \texttt{gender\ ==\ "Female"}. Put all
these posteriors in the same graphic. (One approach is to use the
\textbf{patchwork} package.) The elegant approach is to create a
function which takes a tibble and a restriction, and then returns the
plot.

\end{document}
